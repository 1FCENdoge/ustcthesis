\frontmatter
\begin{abstract}
  摘要内容。
  \ustcsetup{keywords = {关键词}}
\end{abstract}

\begin{abstract*}
  Abstract contents.
  \ustcsetup{keywords* = {keywords}}
\end{abstract*}

\tableofcontents

\begin{notation}
  Notations.
\end{notation}

\mainmatter
\chapter{简介}
Chapter text. \par
Another paragraph.
\section{一级节标题}
Section text.
\subsection{二级节标题}
Subsection text.
\subsubsection{三级节标题}
Subsubsection text.
\paragraph{四级节标题}
Paragraph text.
\subparagraph{五级节标题}
Subparagraph text.

测‘引号’和“双引号”。

Test “double” quotaion ‘marks’.

\chapter{浮动体}
\ustcsetup{cite-style=super}
\cite{knuth86a}

\backmatter
\begin{thebibliography}{1}
\bibitem[Knuth(1986)]{knuth86a}
KNUTH~D~E.
\newblock Computers and typesetting: volume~A\quad The
  {\TeX}book\allowbreak[M].
\newblock Reading, MA, USA: Addison-Wesley, 1986.
\end{thebibliography}

\appendix
\chapter{附录章节}
附录内容。

\begin{acknowledgements}
  致谢内容。
\end{acknowledgements}

\begin{publications}
  发表文章。
\end{publications}
